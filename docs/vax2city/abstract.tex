\renewcommand{\emph}[1]{\textbf{\textsc{\MakeLowercase{#1}}}}
\emph{Background.}
In the current global monkeypox outbreak,
many jurisdictions have been faced with limited vaccine supply,
motivating interest in efficient allocation.
We sought to explore optimal vaccine allocation between two linked transmission networks
over a short-term time horizon, across a range of epidemic conditions.
\emph{Methods.}
We constructed a deterministic compartmental SVEIR model of monkeypox transmission.
We parameterized the model to reflect
two representative, weakly connected \GBMSM sexual networks (cities) in Ontario.
We simulated roll-out of 5000 vaccine doses over 15 days,
starting 60 days after epidemic seeding by 10 imported cases.
Within this model, we varied:
the relative city (network) sizes,
epidemic potentials ($R_0$),
between-city mixing,
and distribution of imported/seed cases between cities.
In each context (combination of varied factors),
we then identified the ``optimal'' allocation of doses between cities
--- resulting in the fewest infections overall by day 120.
\emph{Results.}
Under our modelling assumptions, we found that a fixed supply of vaccines
could generally avert more infections over short-term time horizons when prioritized to:
a larger transmission network,
a network with more initial infections, and
a network with greater $R_0$.
Greater between-city mixing decreased the influence of initial seed cases, and
increased the influence of city $R_0$ on optimal allocation.
Under mixed conditions (e.g. fewer seed cases but greater $R_0$),
optimal allocation saw doses shared between cities,
suggesting that proximity-based and risk-based vaccine prioritization
can work in combination to minimize transmission.
\emph{Interpretation.}
Prioritization of limited vaccine supply based on network-level risk factors
can help minimize transmission during an emerging epidemic.
Such prioritization should be grounded in an understanding of context-specific drivers of risk,
and should acknowledge the potential connectedness of multiple transmission networks.
