\renewcommand{\emph}[1]{\par\textbf{#1}}
\emph{Background.}
The current global monkeypox virus (\MPXV) outbreak
has disproportionately affected gay, bisexual, and other men who have sex with men (\GBMSM).
Given that many jurisdictions have been faced with limited supplies of \MPXV vaccine,
we aimed to explore optimal vaccine allocation between two linked \GBMSM transmission networks
over a short-term time horizon, across several epidemic conditions.
\emph{Methods.}
We constructed a deterministic compartmental \MPXV transmission model.
We parameterized the model to reflect
two representative, partially connected \GBMSM sexual networks (cities),
using 2022 data from Ontario.
We simulated a roll-out of 5000 vaccine doses over 30 days
that started 45 days after epidemic seeding with 10 imported cases.
Within this model, we varied:
the relative city (network) sizes, epidemic potentials ($R_0$), between-city mixing,
and distribution of seed cases between cities.
For each combination of varied factors, we identified the allocation of doses between cities
that maximized infections averted by day 90.
\emph{Results.}
Under our modelling assumptions, we found that a limited \MPXV vaccine supply
could generally avert more early infections when prioritized to:
networks that were larger, had more initial infections, or had greater $R_0$.
Greater between-city mixing decreased the influence of initial seed cases
and increased the influence of city $R_0$ on optimal allocation.
Under mixed conditions (e.g., fewer seed cases but greater $R_0$),
optimal allocation required doses shared between cities.
\emph{Interpretation.}
In the context of the current global \MPXV outbreak, we showed that
prioritization of a limited supply of vaccines based on network-level factors
can help maximize infections averted during an emerging epidemic.
Such priori­tization should be grounded in an understanding of context-specific risk drivers,
and should acknowledge potential connectedness of multiple transmission networks.
