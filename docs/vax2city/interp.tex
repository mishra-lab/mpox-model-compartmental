\section{Interpretation}
We sought to explore how different epidemic conditions
could affect optimal allocation of a fixed supply of \MPXV vaccine
across two partially connected transmission networks (e.g., cities or jurisdictions).
Under our modelling assumptions, we found that:
vaccines could generally avert more infections when prioritized to
a larger network,
a network with more initial infections, and
a network with greater epidemic potential ($R_0$).
\par
Although our study, for simplicity, focused on two partially-connected networks,
it highlights the importance of measuring outcomes for a population overall,
by considering that geographies are comprised of interconnected networks.
That is, while cities across Canada, and globally,
feature important within- and between-city differences
in size and configuration of transmission networks \cite{Scott2015,Gesink2018},
and in access to interventions and services \cite{Millett2012,Hart2021,Doran2021},
these cities ultimately remain connected with respect to transmission,
and cannot be considered in isolation over longer time horizons
\cite{Bogoch2015,Gesink2018,Armstrong2020}.
We grounded the 2 networks as ``cities,'' but the implications would hold
across geographic scope via vaccine allocation across
health units, provinces, or even countries.
\par
Within such interconnected settings,
our findings are consistent with previous studies that showed that
prioritizing limited vaccine supply/resources to communities or settings
with the highest epidemic potential (shaped by density and other features of the contact network)
generally yields the greatest benefit for the population overall
\cite{Garnett2005,Anderson2014,Mishra2021}.
We also identified how key factors,
such as number of imported cases and connections between networks,
shape efficient early vaccine roll-out.
\par
Although our model parameterization reflected \GBMSM sexual networks in Ontario,
our findings have wider implications for vaccine roll-out globally.
The persistent absence of vaccine supply and roll-out in
regions already endemic for \MPXV outbreaks across West and Central Africa,
including (although not yet reported) in the context of \GBMSM and sexual minorities \cite{Manirambona2022},
reflects another failure to uphold principles of equity in global health,
paralleling missed opportunities in achieving \textsc{covid-19} vaccine equity \cite{Yamey2022};
such failures also undermine efforts to control and mitigate \MPXV globally \cite{Zarocostas2022}.
\par
Prioritizing based on risk also requires understanding risk.
Early vaccine roll-out in Ontario started in Toronto,
where cases were already detected, the population size was large, and
rates of bacterial sexually transmitted infections suggested
a potentially denser sexual network and thus, greater epidemic potential \cite{Endo2022}.
Our model implemented differential $R_0$ between cities via contact rates;
however, epidemic potential may also be linked to intervention access,
including access to diagnoses and isolation support \cite{Millett2012,Cevik2021supp}.
Thus, our findings suggest that characterizing the drivers of epidemic potential
across jurisdictions and communities is important,
including participatory, community-based surveillance and research
into the contexts that lead to disproportionate risks at a network-level,
not just at an individual-level \cite{Baral2013,Cevik2021net}.
\subsection{Limitations}
Our study aimed to provide fundamental and generalizable findings
using a broad sensitivity analysis to identify conditions that can shape
optimal short-term vaccine allocation with a limited supply.
As with any modelling study, our results depended on
our modelling assumptions and parameter values;
for some of these, limited data were available.
We did not evaluate population-level benefits balanced with potential adverse effects,
given existing data on high safety with the smallpox vaccines used in Canada \cite{Frey2013,WHO2015}.
\par
We used a simple compartmental model, with only two risk groups;
future work would benefit from more nuanced representations of risk
(e.g., using individual-based sexual network models).
Our study also explored only two representative \GBMSM transmission networks (cities)
with a fixed number of doses.
Incorporation of the wider population, additional transmission networks, or
calibration to observed data on cases, service availability, and vaccine uptake
in specific cities or relevant jurisdictions,
could yield more interesting prioritization findings.
However, we expect that our findings using thest two networks
would apply across multiple networks and conditions.
\par
Finally, we restricted our study to a limited vaccine supply with a fixed rollout approach,
and future research would benefit from exploring the sensitivity of results to
different amounts of finite supply,
time–variant vaccination rate, and
number of imported/seed cases,
as well as timing of vaccine availability in relation to epidemic phase.
