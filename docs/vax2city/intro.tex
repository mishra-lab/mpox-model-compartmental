\section{Introduction}
The emerging outbreak of monkeypox virus (\MPXV) worldwide includes
1,435 cases in Canada as of 2022 October 287 \cite{PHAC2022epi}.
A third-generation replication-deficient smallpox vaccine (Imvamune\rtm) has been
licensed for use against monkeypox and related orthopoxviruses in Canada since 2020,
for the purpose of national security \cite{NACI2022vax}.
Shortly after cases were reported in Canadian cities,
rapid pre-exposure prophylaxis vaccination efforts were started
to help reduce acquisition, infectivity, and disease severity
among communities disproportionately affected by \MPXV, including
gay, bisexual, and other men who have sex with men (\GBMSM) \cite{TPH2022vax,Olson2022}.
However, jurisdictions across Canada and beyond
were faced with a limited local supply of vaccines
during the first few weeks of the \MPXV outbreak.
\par
It is well-established that prioritizing a limited supply of vaccines
to subpopulations with a disproportionately higher transmission risk
(i.e., acquisition and/or transmission at the individual level and/or network levels)
can maximize infections averted \cite{Garnett2005,Mishra2021,Greenhalgh1986,Mylius2008}.
Such networks may have different characteristics that shape
the epidemic potential within the network itself \cite{Weiss2020}.
This potential is often quantified via the basic reproduction number $R_0$,
which reflects the expected number of secondary infections
generated by a person who is infected in a fully susceptible population \cite{EpiNow2}.
A network's connectedness to other networks further shapes
if and how many cases are imported
by the time vaccine allocation decisions and roll-out begin \cite{Keeling2005}.
\par
We sought to explore optimal allocation of a fixed supply of \MPXV vaccine
across two partially connected transmission networks (reflecting jurisdictions)
of \GBMSM (reflecting the community with the most cases of \MPXV infection currently)
under different epidemic conditions.
Specifically, we explored differences between two jurisdictions in:
\GBMSM population size; epidemic potential ($R_0$); imported/seed cases;
and connectedness of the two jurisdictions.
Our goal was to produce fundamental and generalizable insights into
the prioritization of \MPXV vaccine in the context of interconnected sexual networks,
using jurisdictions (cities) within Ontario as an example,
to guide policy-makers in allocating scarce vaccines to maximize infections averted.
