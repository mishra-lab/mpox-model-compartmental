\section{Introduction}
The emerging outbreak of monkeypox (MPXV) virus worlwide includes X cases in Canada as of X, 2022.
Each jurisdiction, across countries and within Canada, are faced with a limited local supply of vaccines 
which are being used to help reduce acquistion, infectivity, and/or modify disease severity. 
Although data on vaccine efficacy against each mechanism (reduction in susceptibility, infectivity, disease severity) 
remains limited at this stage, the pre-emptive use of vaccines within jurisdictions is being used to 
try and limit onward transmission or MPXV spread within communities of gay, bisexual, and other 
men who have sex with men (gbMSM) disproportionately affected by MPXV.
Given the limited supply available to jurisdictions, it is well-established that 
prioritizing a limited supply of vaccines
to sub-populations who experience disproportionately higher 
individual-level and network-level acquisition and transmission risks
can maximize infections averted. % CITE: hotspot, cite other HIV/STI papers including Garnet herd immunity paper, etc
Such networks may have different "densities" or characteristics that shape 
the epidemic potential ($R_0$) of MPXV if it enters the network. A network's connectedness
to other jurisdictions or networks further shape the chances and magnitude of imported cases
by the time vaccine roll-out and allocation decisions begin. For example, within Canada and
its regions, each jurisdiction faced decisions with a limited early supply of an emergent federal 
stockpile of Immavune (R) % cite PHAC vaccine reports, etc.
to prioritize cities for the first few weeks of its vaccine roll-out, pending additional supply.
\par
We sought to explore optimal allocation of a fixed supply of monkeypox vaccine
across two jurisdictions --- i.e. weakly connected transmission networks ---
under different epidemic conditions.
Specifically, we explored differences between two jurisdictions in:
population size of gbMSM; epidemic potential ($R_0$); imported/seed cases;
and connectedness of the two jurisdictions. % change terminology to jurisdictions throughout
The goal of this modeling study was to produce fundamental and generalizable insights into 
MPXV vaccine prioritization in the context of interconnected sexual networks, using ciites (jurisdictions) 
within a province like Ontario, Canada as an example.