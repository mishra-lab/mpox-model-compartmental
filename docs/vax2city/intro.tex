\section{Introduction}
The emerging outbreak of monkeypox virus (\MPXV) worldwide includes
1,112 cases in Canada as of 2022 August 17 \cite{PHAC2022epi}.
A third-generation replication-deficient smallpox vaccine (Imvamune\rtm) has been
licensed for use against monkeypox and related orthopoxviruses in Canada since 2020,
for the purpose of national security \cite{PHAC2022vax}.
Shortly after local cases were reported,
rapid pre-exposure prophylaxis vaccination efforts were initiated
to help reduce acquisition, infectivity, and/or disease severity
among communities disproportionately affected by \MPXV, including
gay, bisexual, and other men who have sex with men (\GBMSM) \cite{TPH2022vax}.
However, many jurisdictions, across countries and within Canada,
were faced with a limited local supply of vaccines
during the first few weeks of \MPXV outbreak.
\par
It is well-established that prioritizing a limited supply of vaccines
to sub-populations experiencing disproportionately higher risk
--- individual-level and/or network-level acquisition and/or transmission risk ---
can maximize infections averted \cite{Garnett2005,Mishra2021}.
Such networks may have different characteristics that shape
the epidemic potential ($R_0$) within the network itself.
A network's connectedness to other networks further shapes
the chances and number of imported cases
by the time vaccine allocation decisions and roll-out begin.
\par
We sought to explore optimal allocation of a fixed supply of \MPXV vaccine
across two jurisdictions --- i.e. weakly connected transmission networks ---
under different epidemic conditions.
Specifically, we explored differences between two jurisdictions in:
population size of \GBMSM; epidemic potential ($R_0$); imported/seed cases;
and connectedness of the two jurisdictions.
The goal of this modeling study was to produce fundamental and generalizable insights into
\MPXV vaccine prioritization in the context of interconnected sexual networks,
using jurisdictions (cities) within a province like Ontario, Canada as an example.
