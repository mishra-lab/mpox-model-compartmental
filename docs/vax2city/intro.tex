\section{Introduction}
The emerging outbreak of monkeypox virus (\MPXV) worlwide includes
1,059 cases in Canada as of 2022 August 12 \cite{PHAC2022epi}.
Many jurisdictions, across countries and within Canada,
are faced with a limited local supply of vaccines,
which are being used to help reduce acquistion, infectivity, and/or disease severity.
Although data on vaccine efficacy against each aspect (acquistion, infectivity, disease severity)
remains limited at this stage \cite{Fine1988,PHAC2022imvamune,CDC2022vax},
pre-emptive vaccination is being used to try to limit \MPXV spread
within communities of gay, bisexual, and other men who have sex with men (\GBMSM),
who are disproportionately affected by \MPXV.
\par
It is well-established that prioritizing a limited supply of vaccines
to sub-populations experiencing disproportionately higher risk
--- individual-level and/or network-level acquisition and/or transmission risk ---
can maximize infections averted \cite{Garnett2005,Mishra2021}.
% CITE: hotspot, cite other HIV/STI papers including Garnet herd immunity paper, etc
Such networks may have different ``densities'' or characteristics that shape
the epidemic potential ($R_0$) of \MPXV, if it enters the network.
A network's connectedness to other networks further shapes
the chances and number of imported cases
by the time vaccine allocation decisions and roll-out begin.
For example, within Canada and its regions, many jurisdictions
faced decisions about how to prioritize allocation of
a limited emergency federal stockpile of Immavune\rtm \cite{PHAC2022imvamune}
during the first few weeks of \MPXV outbreak, pending additional supply.
\par
We sought to explore optimal allocation of a fixed supply of \MPXV vaccine
across two jurisdictions --- i.e. weakly connected transmission networks ---
under different epidemic conditions.
Specifically, we explored differences between two jurisdictions in:
population size of \GBMSM; epidemic potential ($R_0$); imported/seed cases;
and connectedness of the two jurisdictions.
The goal of this modeling study was to produce fundamental and generalizable insights into
\MPXV vaccine prioritization in the context of interconnected sexual networks,
using jurisdictions (cities) within a province like Ontario, Canada as an example.