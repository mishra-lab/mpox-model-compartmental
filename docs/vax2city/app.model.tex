\section{Model Details}\label{app.model}
Table~\ref{tab:notation} summarizes the notation used,
and Figure~\ref{fig:model} illustrates the model structure,
repeated (verbatim) in Figure~\ref{fig:app.model} for easier reference.
\begin{table}[h]
  \centering
  \vskip-1ex % HACK
  \caption{Notation}
  \label{tab:notation}
  \begin{tabular}{cl}
  \toprule
     Symbol     & Definition                                                \\
  \midrule
       $c$      & city index $\in$\{A, B\}                                  \\
       $r$      & risk group index $\in$ \{high, low\}                      \\
       $y$      & type of contact index $\in$ \{sexual, close non-sexual\}  \\[1ex]
       $N$      & population size                                           \\
       $C$      & contact rate                                              \\
       $Q$      & total contacts offered: $NC$                              \\
   $\epsilon$   & assortativity parameter $\in$ [1: assortative, 0: random] \\[1ex]
    $\lambda$   & incidence rate (force of infection)                       \\
     $\beta$    & secondary attack rate\tn{a}                               \\
  $\sigma^{-1}$ & duration of latent/incubation period                      \\
  $\gamma^{-1}$ & duration of infectious/symptom period                     \\
     $\Phi$     & probability of contact formation                          \\[1ex]
     $\rho$     & proportion isolating among infectious                     \\
      $\nu$     & vaccination rate                                          \\
       $f$      & vaccine effectiveness (leaky=type)                        \\
  \bottomrule
\end{tabular}
\floatfoot
All durations in days; all rates in per-day.
\tnt[a]{per-partnership transmission probability}
  \vskip-2ex % HACK
\end{table}
\begin{figure}
  \begin{subfigure}[b]{0.55\linewidth}
    \centerline{\includegraphics[scale=1]{model.health}}
    \vskip3em
    \caption{Health states and transitions}
    \label{fig:app.model.health}
  \end{subfigure}\hfill
  \begin{subfigure}[b]{0.55\linewidth}
    \centerline{\includegraphics[scale=1]{model.city.risk}}
    \caption{Cities, risk groups, and contact networks}
    \label{fig:app.model.city.risk}
  \end{subfigure}
  \caption{Model structure}
  \label{fig:app.model}
  \floatfoot
  (\subref{fig:app.model.health})
  $S$:~susceptible;
  $V$:~vaccinated;
  $E$:~exposed;
  $I$:~infectious;
  $H$:~isolating;
  $R$:~recovered.
  (\subref{fig:app.model.city.risk})
  Arrow opacity is qualitatively related to the chance of sexual contact formation from any group to another.
  See Table~\ref{tab:notation} for rate definitions.
\end{figure}
\subsection{Differential Equations}\label{app.model.eqn}
Equation~\eqref{eq:ode} summarizes the system of differential equations for the health states;
each equation is repeated for each combination of city~$c$ (A,~B) and risk group~$r$ (higher,~lower) (4~total),
but we omit the $cr$ index notation for clarity.
\begin{subequations}\label{eq:ode}
  \begin{alignat}{5}
    \frac{d}{dt}&S   &&= - \nu S - \lambda S \\
    \frac{d}{dt}&V   &&= + \nu S - (1 - f) \lambda V \\
    \frac{d}{dt}&E   &&= + \lambda S + (1-f) \lambda V - \sigma E\\
    \frac{d}{dt}&\,I &&= + \sigma E - \frac{\gamma}{1-\rho}\,I \\
    \frac{d}{dt}&H   &&= + \frac{\gamma}{1-\rho}\,I - \frac{\gamma}{\rho} H\\
    \frac{d}{dt}&R   &&= + \frac{\gamma}{\rho} H
  \end{alignat}
\end{subequations}
\subsection{Incidence Rate}\label{app.model.inc}
The incidence rate (force of infection) for
non-vaccinated susceptible individuals in city $c$ and risk group $r$ (``group $cr$'') is defined as:%
\footnote{The online Appendix to the published paper erroneously includes a term ``$(1 - \rho)$''
  in \eqref{eq:foi}, which should have been removed after the isolating state $H$
  was added during revisions.}
\begin{equation}\label{eq:foi}
  \lambda_{cr} = \sum_{yc'r'} \beta_{y}\,C_{ycr}\,\Phi_{ycrc'r'} \frac{I_{c'r'}}{N_{c'r'}}
\end{equation}
where:
$\beta_{y}$ is the transmission probability per type-$y$ contact;
$C_{ycr}$ is the type-$y$ contact rate among group $cr$;
$\Phi_{ycrc'r'}$ is the probability of forming a type-$y$ contact
  with group $c'r'$ (contacts) among group $cr$ (self); and
$N_{cr}$ is the size of group $cr$.
\par
Among vaccinated, the incidence rate is simply reduced by a factor $(1-f)$, where
$f$ is the vaccine effectiveness (leaky-type).
\subsection{Mixing}\label{app.model.mix}
Mixing between risk groups and cities was implemented using
an adaptation of a common approach \cite{Nold1980,Garnett1994}.
We denote the total contacts ``offered'' by group~$cr$ as: $Q_{cr} = N_{cr} C_{cr}$;
and denote the margins (sums) $Q_{c} = \sum_{r}Q_{cr}$; $Q_{r} = \sum_{c} Q_{cr}$; and $Q = \sum_{cr}Q_{cr}$.
The probability of contact formation with group $c'r'$ among group $cr$ is defined as:
\begin{equation}
  \Phi_{crc'r'} =
    \epsilon_c \delta_{cc'} \left(
      \epsilon_r \delta_{rr'} + (1 - \epsilon_r) \frac{Q_{c'r'}}{Q_{c'}}
    \right) +
    (1 - \epsilon_{c}) \frac{Q_{c'}}{Q} \left(
      \epsilon_r \delta_{rr'} + (1 - \epsilon_r) \frac{Q_{r'}}{Q}
    \right)
\end{equation}
where:
$\delta_{ii'} = \{1~\text{if}~i = i'; 0~\text{if}~i \ne i'\}$ is an identity matrix; and
$\epsilon_{c}, \epsilon_{r} \in [0,1]$ are assortativity parameters
for mixing among cities and risk groups, respectively, such that
$\epsilon = 1$ yields complete group separation and
$\epsilon = 0$ yields completely random (proportionate) mixing.
For clarity, we omit the index of contact type $y$,
although $\epsilon_r$, $C_{cr}$ and thus $\Phi_{crc'r'}$ are all further stratified by $y$.
\subsection{City $R_0$}\label{app.model.R0}
The basic reproduction number $R_0$ for each city was defined
in the absence of vaccination and ignoring between-city mixing --- i.e., with $\epsilon_c = 1$.
Following \cite{Diekmann1990}, we define $R_0$ as
the dominant eigenvalue of the city-specific next generation matrix $K$;
matrix elements $K_{rr'}$ are defined as:
\begin{equation}
  K_{rr'} = (1 - \rho) \sum_{y} \beta_{y} C_{yr} \Phi_{yrr'} \frac{N_{r}} {N_{r'}} \gamma^{-1}
\end{equation}
where:
$\rho$ is the proportion isolating among infectious;
$\beta_{y}$ is the transmission probability per type-$y$ contact;
$C_{yr}$ is the type-$y$ contact rate among group $r$;
$\Phi_{yrr'}$ is the probability of type-$y$ contact formation with group $r'$ among group $r$;
$N_{r}$ is the size of group $r$; and
$\gamma^{-1}$ is the duration of infectiousness.
\subsection{Vaccine Allocation}\label{app.model.vax}
Vaccination is modelled as distribution of 5000 doses over 30 days from day 45 (167 doses per day).
Vaccines are prioritized to the high risk group with 90\% sensitivity, such that
4500 doses actually reach the high risk group, and
500 doses are given to the lower risk group.
Figure~\ref{fig:prev.vax} illustrates vaccination coverage/counts by city/risk group
for an example allocation of 80\% to city~A and 20\% to city~B.
\begin{figure}[h]
  \includegraphics[width=\linewidth]{vax}
  \caption{Example vaccine allocation: 80\% to city~A, and 90\% to high risk group}
  \label{fig:prev.vax}
  \floatfoot
  Gray bar indicates period of vaccine roll-out (days 45--75)
\end{figure}
\subsection{Parameterization}\label{app.model.param}
Model parameter values and stratifications are summarized in Table~1,
repeated (verbatim) in Table~\ref{tab:model.params.app} for easier reference.
\begin{table}
  \centering
  \caption{Model parameters, including default values and ranges explored via grid sweep}
  \label{tab:model.params.app}
  \small
\begin{tabular}{llrcc}
  \toprule
  Parameter                          & Stratum                   &      Value &      Range       & Ref \\
  \midrule
  Population size                    & overall                   &    100,000 &                  & \cite{Wang2021}\tn{a} \\
                                     & fraction in city A        &        .50 &    [.20,~.80]    & \tn{a} \\
  Fraction higher risk               & city A                    &        .10 & [.01,~.50]\tn{b} & \cite{Wang2021}\tn{a} \\
                                     & city B                    &        .10 &                  & \cite{Wang2021}\tn{a} \\[1ex]
  Contact rate                       & close non-sexual, all     &          1 &                  & \tn{a} \\
                                     & sexual, low risk          &        .01 &                  & \cite{Wang2021}\tn{a} \\
                                     & sexual, high risk, city A & .178\tn{c} & [.10,~.25]\tn{b} & \cite{Wang2021,Endo2022}\tn{a} \\
                                     & sexual, high risk, city B & .178\tn{c} &                  & \cite{Wang2021,Endo2022}\tn{a} \\[1ex]
  Assortativity                      & cities, all contacts      &        .90 &    [.70,~1.0]    & \cite{Armstrong2020}\tn{a} \\
                                     & risk, close non-sexual    &          0 &                  & \tn{a} \\
                                     & risk, sexual              &        .50 &                  & \tn{a} \\
  Per-contact \SAR                   & close non-sexual          &        .05 &                  & \cite{Beer2019} \\
                                     & sexual                    &  .90\tn{c} &                  & \cite{Endo2022}\tn{a} \\[1ex]
  Initial infections                 & overall                   &         10 &                  & \tn{a} \\
                                     & fraction in city A        &        .50 &    [0.0,~1.0]    & \tn{a} \\[1ex]
  Duration of period                 & latent/incubation         &          7 &                  & \cite{Charniga2022,Thornhill2022,Miura2022} \\
                                     & infectious/symptoms       &         21 &                  & \cite{Adler2022,Thornhill2022} \\
  Fraction isolated among infected   &                           &        .50 &                  & \cite{Thornhill2022}\tn{a} \\[1ex]
  Vaccines available                 &                           &       5000 &                  & \tn{a} \\
  Vaccine effectiveness\tn{d}        &                           &        .85 &                  & \cite{Fine1988,PHAC2022vax,CDC2022vax} \\
  Vaccine prioritization sensitivity & high risk                 &        .90 &                  & \cite{TPH2022vax}\tn{a} \\
  Vaccine allocation                 & city A                    &        .50 & [0.0,~1.0]\tn{e} & --- \\
  \bottomrule
\end{tabular}
\floatfoot
All durations in days; all rates in per-day.
\SAR: secondary attack rate.
\tnt[a]{Assumed / representative.}
\tnt[b]{Calculated to fit $R_0 \in [1,2]$.}
\tnt[c]{Calculated to fit $R_0 = 1.5$,
  reflecting pre-vaccination estimate of \MPXV $R_0$ in Ontario \cite{PHO2022ont}
  via \cite{EpiNow2}.}
\tnt[d]{Leaky-type.}
\tnt[e]{Optimized parameter.}
\end{table}
\paragraph{Risk Groups and Sexual Contacts}
Parameterization of risk groups and contacts was primarily informed by existing analyses conducted
to support mathematical modelling of \textsc{hiv}-transmission among \GBMSM in Canada
\cite[n.b. Appendix~3.2]{Wang2021},
since sexual history data among \GBMSM diagnosed with \MPXV in Canada are not yet available.
These analyses stratified \GBMSM into
88--94\% lower risk, with on average 4 sexual partners per-year ($\approx .01$ per day), and
6--12\% higher risk, with approximately 6-times as many partners ($\approx .07$ per day),
reflecting common stratification corresponding to
rates of bacterial STI and partner number distributions \cite{Hart2021,Brogan2019}.
Our present model includes even greater partner numbers among the higher risk group (.10--.25 per day),
partly to fit \MPXV $R_0 \in [1,~2]$, and because
the 6-fold value in \cite{Wang2021} was mainly applied as a generalized proxy for
6-times higher HIV incidence.
Weighted pooling of data from three studies \cite{MaleCall2013,Lachowsky2016,Wilton2016}
suggested that approximately 12\% of respondents
reported 20+ sexual partners in the past 6 months ($\approx .11+$ per day).
Our \MPXV model also models transmission risk per-partnership,
versus per-contact (sex act) as in \cite{Wang2021};
with high \SAR, \MPXV transmission risk would be expected to be
driven more by numbers of partners than by total contacts (sex acts).
\par
A retrospective and rapid sexual history survey of
45 individuals diagnosed with \MPXV identified that
60\% (27 of 45) were diagnosed with an STI in the previous year,
44\% (20 of 45) reported more than 10 sexual partners in the previous 3 months, and
44\% (20 of 45) reported group sex during the incubation period~\cite{UKHSA2022}.
\paragraph{Close, Non-Sexual Contacts}
We defined close, non-sexual contacts as
direct exposure of broken skin or mucous membranes,
or to bodily fluids or potentially infectious material (including clothing or bedding)
without appropriate personal protective equipment,
such as sleeping in the same bed.
Based on available data on types of partnerships,
30--60\% of \GBMSM in Canada report a regular sex partner \cite{Milwid2022},
and data on additional living conditions (such as cohabitating with non-sexual partners) was not available.
\paragraph{Network Connectivity}
There is limited data on proportion of contacts (sexual and close non-sexual)
formed between different regional \GBMSM networks,
Such proportions will also depend on the geographic scale of the networks considered,
while our study aimed to be generalizable across scales.
In \cite{Armstrong2020} 37.5\% of 269 respondents from Waterloo, Ontario
had travelled outside the region for sex;
however, this does not necessarily reflect the proportion of all sex outside the region.
From limited case-series data, evidence suggests that
a smaller fraction have likely acquired \MPXV infection via sex in other cities:
among cases among Toronto residents seen at Unity Health Toronto between 2022 May 20 and July 15:
2/27 were identified as infection from sexual exposures outside Toronto
\cite[personal communication]{Mishra2022}.
\paragraph{Monkeypox Virus (\MPXV) \& Reproduction Number}
Updated epidemiological data on \MPXV infection and transmission
in the context of the present epidemic are rapidly emerging \cite{Thornhill2022,PHO2022synth}.
In the absence of high-quality evidence on
the secondary attack rate (\SAR) of sexual transmission,
we assumed a relatively high \SAR of 0.9 (per-partnership),
drawing on local patient histories, and in order to reproduce $R_0 \in [1,2]$.
We estimated $R_0 \in [1,2]$ using \MPXV case data from Ontario \cite{PHO2022ont}
before widespread vaccine roll-out (2022 May 13 -- July 4)
using the \texttt{EpiNow2} R package \cite{EpiNow2}.
We further calibrated the \SAR for close, non-sexual to reproduce
approximately 95\% incidence via sexual vs close, non-sexual contacts \cite{Vaughan2022}.
\par
In another model \cite{Endo2022},
the modelled $R_0$ for a \GBMSM sexual network was greater, even for smaller \SAR.
Two main factors may explain this discrepancy in
modelled $R_0$ vs \SAR in \cite{Endo2022} vs our model.
First, isolation was not explicitly modelled in \cite{Endo2022};
thus the reported \SAR in \cite{Endo2022} can be considered as
after considering isolation, i.e., reduced.
Second, the branching process model in \cite{Endo2022}
captured greater risk heterogeneity than our model,
and focused especially on capturing the highest levels of risk (``heavy tail'').
Such heterogeneity is directly related to $R_0$
through the coefficient of variation in contact rates \cite{Anderson1986}.
Thus, this difference in model structure could further explain why
modelled $R_0$ would be greater in \cite{Endo2022}, for even similar \SAR.
Finally, our aim was to obtain generalizable insights about network-level vaccine prioritization,
rather than to model specific contexts within Ontario;
as such, we do not expect our main findings to change
with moderate changes to the model simplifications regarding transmission.
