\section{Discussion}  %Jesse - I really like this first para of discussion as some text to send to Betty if you can send that to Betty?
We sought to explore how different epidemic conditions
could affect optimal allocation of a fixed supply of monkeypox vaccine
across two weakly connected transmission networks (e.g. cities or jurisdictions).
Under our modelling assumptions, we found that
vaccines could generally avert more infections when prioritized to
a larger network,
a network with more initial infections, and
a network with greater epidemic potential $R_0$
defined by contact rates in a higher risk group.
\par
Although the study, for simplicity, centered on two interconnected networks, it highlights the 
importance of measuring outcomes for an overall population by considering that a geography is comprised
of interconnected networks (within its jurisdictional boundaries and between jurisdictions/cities). 
That is, cities like Toronto and other across the province and Canada, and globally, are all connected; 
alongside within-city differences in size and configuration of sexual networks and access to interventions/services. % refs
In the first month following emergence of MPXV in Ontario, the province began its initial limited vaccine roll-out to 
jurisdictions that had already detected cases, which was in Toronto - a jurisdiction that has the largest population size, 
and some features (such as rates of bacterial sexually transmitted infections as a proxy) that suggest a potentially more dense sexual network, and greater epidemic potential for MPXV. % cite LHSTM paper re: dense/configuratin of sexual network given their use of degree deistribution with branching rocess model and MPXV
Of note, although our model explicitly simulated epidemic potential on the basis of contact rates, access to interventions (for example, access to diagnoses and diagnostics, isolation support, etc.) can also shape epidemic potential. % cite
Thus, the findings signal the importance of characterizing the "why" of epidemic potential across jurisdictions and communities, including participatory, community-based surveillance and research into the contexts that lead to disproportioante risks at a network-level (not just an individual-level). %cite: the nature perspective paper by Muge Cevik and Baral; Baral et al socio-ecological model of HIV, etc.
\par
Our findings are consistent with previous studies that confirm that prioritizing limited vaccine 
supply/resources to communities or settings with the highest epidemic potential (shaped by 
density and features of the contact network) genrally yield the greatest overall benefit for the 
overall population. We also identified how other key factors, such as volume of imported cases and 
interconnections between networks shape the optimal early vaccine roll-out. Although the study was situated
in the context of GBMSM sexual networks in Ontario as an example, the findings have wider implications for vaccine roll-out globally. The absence of systematic vaccine supply and roll-out in countries and sub-national regions already endemic for MPXV outbreaks across West and Central Africa, and including (although not yet reported) in the context of GBMSM and sexual minorities, poses the largest threat to the control and mitigation of MPXV globally. %cite WHO, GAVI, etc. 
\par
\paragraph{Limitations}
Our study was meant to provide fundametnal and generalizable findings, and thus
restricted to a broad sensitivity analyses to idnetify the conditions that can shape 
optimal vaccine allocation in a limited time, with very limited supply. One limitiation 
of our study is that we used simple, compartmental models, with 2 dimensions of heterogeneity only; and 
future work woudl benefit from individual-based sexual network models. Second, our study only examined
two jurisdictions or networks, and additional stratifications (more jurisdictions/cities) could refine allocation further. 
However, the general principles and insights from two jurisdictions would apply across multiple jurisdictions. 
\par
% conclusion  TODO