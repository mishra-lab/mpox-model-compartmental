\section{Interpretation}
We sought to explore how different epidemic conditions
could affect optimal allocation of a fixed supply of monkeypox virus (\MPXV) vaccine
across two weakly connected transmission networks (e.g. cities or jurisdictions).
Under our modelling assumptions, we found that:
vaccines could generally avert more infections when prioritized to
a larger network,
a network with more initial infections, and
a network with greater epidemic potential ($R_0$).
\par
Although our study, for simplicity, focused on two weakly-connected networks,
it highlights the importance of measuring outcomes for a population overall,
by considering that geographies are comprised of interconnected networks.
That is, while cities across Canada, and globally,
feature important within- and between-city differences
in size and configuration of transmission networks \cite{},
and in access to interventions/services \cite{},
ultimately these cities remain connected with respect to transmission,
and cannot be considered in isolation over longer time horizons \cite{}.
\par
Within such interconnected settings,
our findings are consistent with previous studies which show that
prioritizing limited vaccine supply/resources to communities or settings
with the highest epidemic potential (shaped by density and other features of the contact network)
generally yields the greatest benefit for the population overall
\cite{Garnett2005,Anderson2014,Mishra2021}.
We also identified how key factors,
such as number of imported cases and connections between networks,
shape efficient early vaccine roll-out.
While our model parameterization reflected \GBMSM sexual networks in Ontario,
our findings have wider implications for vaccine roll-out globally.
The persistent absence of vaccine supply and roll-out in
regions already endemic for \MPXV outbreaks across West and Central Africa,
including (although not yet reported) in the context of \GBMSM and sexual minorities,
poses the largest threat to the control and mitigation of \MPXV globally~\cite{}.
% CITE WHO, GAVI, etc.
\par
Prioritizing based on risk also requires understanding risk.
Early vaccine roll-out in Ontario reached Toronto,
where cases were already detected, the population size was large, and
rates of bacterial sexually transmitted infections suggested
a potentially denser sexual network and thus, greater epidemic potential \cite{Endo2022}.
Our model implemented differential $R_0$ between cities via contact rates;
however, epidemic potential may also be linked to intervention access,
including access to diagnoses and isolation support \cite{}.
Thus, our findings signal the importance of characterizing
the drivers of epidemic potential across jurisdictions and communities,
including participatory, community-based surveillance and research
into the contexts that lead to disproportioante risks at a network-level,
not just an individual-level \cite{Baral2013,Cevik2021}.
% CITE: the nature perspective paper by Muge Cevik and Baral; Baral et al socio-ecological model of HIV, etc.
\par
Our study aimed to provide fundamental and generalizable findings, and thus
explored a broad sensitivity analysis to identify conditions that can shape
optimal short-term vaccine allocation, with very limited supply.
One limitiation of our study is that
we used a simple compartmental model, with only two risk groups;
future work would benefit from more nuanced representations of risk,
for example, using individual-based sexual network models.
Second, our study only examined two transmission networks (``cities'');
incorporation of additional networks could yield more interesting prioritization findings.
However, we expect that the general principles and insights from two networks
would apply across multiple networks.
\par
Conclusion [TODO]